\documentclass{article}
\usepackage{graphicx} % Required for inserting images
\usepackage{amsmath}
\usepackage{amsfonts}
\usepackage{amssymb}
\usepackage{amsthm}
\usepackage{braket}

\usepackage{bbold} %for id

\usepackage{xcolor}
\usepackage[margin = 1.5cm]{geometry}
\setlength{\parskip}{1.5ex}
\def\baselinestretch{1.4}
\pagestyle{plain}


\usepackage{color}
\definecolor{darkblue}{rgb}{0,0,0.6}
\definecolor{purple}{rgb}{0.4,.2,0.7}
\definecolor{darkgreen}{rgb}{0,0.5,0}
\newcommand\AM[1]{{\it \color{darkblue}  [#1 - AM]}}

\usepackage[bookmarks = false, colorlinks = true, linkcolor = darkblue, citecolor = purple]{hyperref}


\newtheorem{conn}{Connection}
\renewcommand{\i}{\mathrm{i}}
\renewcommand{\d}{\mathrm{d}}
\title{Condensed Matter Diary}
\author{}
\date{\today}


\renewcommand{\i}{\mathrm{i}}


\begin{document}

\maketitle
\begin{abstract}
    This is just some notes I took while reading books, papers/watching lectures/doing some calculations. The purpose is just to force myself to TeX out the scrap notes I take while doing research. I also want to make it public so that I remain motivated to keep updating it. Hopefully this selfish endeavour will be useful to the curious eyes that come across it :)\\
    \textbf{Disclaimer:} Pretty much everything in here is either directly copied or lightly rephrased versions of what I came across and original sources are listed whenever relevant.
\end{abstract}
List of topics to start writing about:
\begin{enumerate}
    \item LSM, LR theorems
    \item Haldane phase/conjecture, AFM O(3) NLSM paper.
    \item AKLT model and properties. Identify the VBS ground state as an MPS state.
    \item BLBQ models in general (Affleck's paper) understand the phase diagram and in particular the ULS model (or SU(3) point) and Takhtajan, Fateev, Zamolodchikov model. Understand the mapping to WZW theory. Understand the yang-baxter/integrability pov on these models.
    \item Connection between spectrum, correlator's behaviour and order/disorder. Also on LRO vs SSB (mermin-wagner).
    \item KT transformation and connection to our mapping.
    \item Dualities (duality webs-witten, senthil)(boson vortex duality, flux threading/attachment)
    \item Krammers Wannier, Jordan Wigner iron out some details.
    \item Edge modes (gapping the bulk) (FQH-LL bulk boundary), Kitaev model+Majoranas
    \item SPT phases(Projective reps/group cohomology pov- xie chen, xiao gang wen...) [see mcgreevy lecs]
    \item Generalised Symmetries
    \item Bosonization done right
    \item QFT in 1+1d - David tong gauge theory notes
    \item QFT in 2+1 d- David tong+Zohar lecs
    \item G theorem, boundary RG flow
    \item Hastings entropy Area law for many body , boundary term for topological order-mutual information
    \item Anomalies
    \item Disordered systems, localization stuff. Check to see if any of the old notes are still around for an extensive topics list. Else see reading list email/message.
\end{enumerate}
\section{Coherent states, segal bargmann transform, heat kernel $\leftrightarrow$ oscillator propagator}
Some stuff to look at: \cite{PhysRevD.29.1107}

\section{Lieb-Schultz-Mattis Theorem}
(Source: Hal Tasaki's lecture)


Spin rotation operator (unitary), rotates the spin state by theta about the $\alpha$ axis.
$\exp{[-i\theta S^{\alpha}]}$.

LSM theorem is a no-go theorem which states that certain quantum many-body systems cannot have a unique ground state with a nonzero energy gap.
This statement is interesting because it shows that symmetry of a quantum spin system places a strong constraint on its low energy properties.

The original theorem was in the context of the antiferromagnetic heisenberg chain. Consider a 1d lattice of quantum spins described by the Hamiltonian
$\hat{H} = \sum^{L}_{j=1} \hat{S}_j \cdot \hat{S}_{j+1}$, with ground state energy $E_{\text{gs}}$, and with periodic boundary conditions.
For half integer spins$\left(S = \frac{1}{2}, \frac{3}{2}, \cdots\right)$, there exists an energy eigenvalue $E$ such that
\begin{equation}
    E_{\text{gs}} \leq E \leq E_{\text{gs}} + \frac{C}{l}
\end{equation}
for any $l<L$ and some constant $C = 8\pi^2S^2$. This not only tells us that $E$ is an excitation energy that is larger than the ground state energy, but since $l$ can be made as arbitrarily small as you want, this also suggests that the system is gapless!

Outline of the proof
\begin{enumerate}
    \item Invoke the Marshall 1955, Lieb, Mattis 1962 theorem, which states that the ground state $\ket{\text{GS}}$ of $\hat{H}$ is unique $\forall L$ even (finite).
    \item Make a variational estimate, exploiting the above uniqueness. Using the $U(1)$ rotational invariance of $\hat{H}$, we can conclude that the ground state has to be rotationally invariant (As it is unique), that is
    \begin{equation}
        \exp{\left[-\i \sum^{L}_{j=1} \theta_j \hat{S}^z_j\right]} \ket{GS} = \ket{\text{GS}}
    \end{equation}
    We can then do a gradual \emph{non-uniform} (site-dependent) rotation or a twist given by
    \begin{equation}
        \hat{U}_l = \exp{\left[-\i \sum^{l}_{j=1} \theta_j \hat{S}^z_j\right]}, \quad \theta_j = \frac{2\pi}{l} j= \Delta\theta\cdot j ,~ 0\leq j \leq l
    \end{equation}
    Note that $\theta_j = 2\pi \text{ for }  j \geq l,~ \theta_j = 0 \text{ for } j\leq 0$, but of course $2\pi = 0$ in this case.
    We use this twist operator to construct our variational state $\ket{\Psi_l} = \hat{U}_l\ket{\text{GS}}$. For $0 \leq j \leq l$ we have
    \begin{equation}
        \braket{\Psi_l| \hat{S}_j \cdot \hat{S}_{j+1}| \Psi_l} = \frac{E_{\text{gs}}}{L} + \mathcal{O}\left(\left(\Delta \theta\right)^2\right)
    \end{equation}
    This tells us that within the region, the local energy density given by LHS is equal to the ground state energy density $\frac{E_{\text{gs}}}{L}$ plus some corrections that go as $\Delta \theta ^2 or \frac{1}{l^2}$. Thus, multiplying by l, $\braket{\Psi_l | \hat{H} | \Psi_l} - E_{\text{gs}} \leq \frac{C}{l}$. This bound is valid for any spin $S$.
    \item Orthogonality of the twisted state.
    We now need to show that this twisted state is different from the ground state.
    We can define a Unitary $\hat{R}$ such that
    \begin{equation}
       \hat{R}\hat{S}^\alpha_j\hat{R} = \begin{cases}
           \hat{S}^\alpha_{l-j} & \alpha = x \\
           -\hat{S}^\alpha_{l-j} & \alpha = y, z
       \end{cases}     
    \end{equation}
    This operator allows us to verify \AM{Not yet understood!!} that $\braket{\text{GS}|\Psi_l}$ is zero for half-integer spins.
\end{enumerate}
In the infinite volume limit, it turns out that there is still a unique ground state with gapless excitations, and not multiple ground states instead.
This theorem also provides no information for the integer spin case. In fact, the integer spin models have a unique ground state with a gap! This is related to the Haldane conjecture/gap.
\subsection{LSM theorem for models with only discrete symmetries}
$U(1)$ invariance is essential for the previous proof. Can we do better?

It turns out that for quantum spin chains with half-integer spins and with a short-ranged translation invariant Hamiltonian that is invariant under time-reversal symmetry i.e. $\hat{S}^{\alpha}_j \to  -\hat{S}^{\alpha}_j,~ \alpha = x, y, z$ it can never be the case that the infinite volume ground state is unique and accompanied by a non-zero gap.

The proof by Y.Ogata, Y.Tachikawa and H.Tasaki relies on using the Ogata index for edge states, developed for the study of SPT phases to obtain a necessary condition for the existence of a unique gapped ground state.
\section{Heisenberg model}
Here is some stuff to read about this (from Fradkin chapter 5)

\begin{enumerate}
    \item Bethe ansatz solution for the ground and excited states.
    \item Mapping to the sine-gordon theory.
    \item non-abelian bosonization
    \item Mapping to the sigma model
\end{enumerate}
Continuum description for large S by Haldane: \cite{PhysRevLett.50.1153}.
The mappings provide us with non-perturbative probes of the ground and excited states. Read the refs (witten, polyakov and wiegmann, affleck)
\section{BLBQ models}
\begin{equation}
    H_{\text{BLBQ}} \left(\theta\right) = \sum_j \cos{\theta} \left(\vec{S}_j\cdot \vec{S}_{j+1}\right) + \sin{\theta} \left(\vec{S}_j\cdot \vec{S}_{j+1}\right)^2
\end{equation}
\subsection{AKLT point}
(Source: Hal Tasaki's Textbook)

AKLT (Affleck, Kennedy, Lieb, Tasaki) is a special point in the class of Bilinear Biquadratic (BLBQ) models and shows up at the $\theta = \arctan{\frac{1}{3}}$ point. At this point, the model has a few properties.
\begin{enumerate}
    \item We only have SO(3) or spin rotation symmetry. 
    \item The ground state is the Valence Bond Solid state. One can derive this by first re-writing the BLBQ Hamiltonian at the AKLT point in terms of the spin 2 projection operator. Depending on the type of boundary conditions imposed, we can have gapless edge modes. For open boundary conditions, the ground state will have 4 gapless edge modes and with periodic boundary conditions there are none. The VBS state can also be written as a Matrix product state \AM{write about this please}
    \item The correlators (spin correlation functions) decay exponentially, indicating a massive excitation. This mass gap or energy gap is in line with Haldane's conjecture, which states that we should see a gapped ground state for integer spin nearest neighbour spin models in 1+1 D.
    \item $\mathbb{Z}_2 \times \mathbb{Z}_2$ is the center and the gapped/SPT properties come from it.
    \item  It is integrable. (Check Bethe-Ansatz/Yang-Baxter) \AM{What does this give you?}
\end{enumerate}

Spin 1 chains also have this decomposition in terms of creation/annihilation ops of U(1), and they have some connection to the chains having twisted boundary conditions.\AM{Vaguely recall this discussion, make it precise}
\subsection{ULS point}
To study the (global?) symmetry of these spin chains, it can be useful to convert to the Abrikosov fermion representation. Since we are working with the spin -1 model, we use 3 fermion channels at each site with a local fermion number constraint $\sum^{3}_{\mu = 1} c^{\dagger}_{j, \mu} c_{j, \mu} = 1.$ \AM{Show how there is SU(2) when the bond/pairing term is present}
For a generic coupling coupling $\gamma$ the model looks like:
\begin{equation}
    H_{\text{BLBQ}}(\gamma) = \sum_{j} c^{\dagger}_{j, \alpha} c_{j, \beta} ~c^{\dagger}_{j+1, \beta}c_{j+1, \alpha} + (\gamma -1)  c^{\dagger}_{j, \alpha} c_{j, \beta} c^{\dagger}_{j+1, \alpha} c_{j, \beta},
\end{equation}
where, the first term describes a hopping between channels at adjacent sites and the second term describes a ``single-bond" pairing between adjacent sites.
 At $\gamma = \frac{4}{3}$ we are at the AKLT point, and at $\gamma = 1$ we are at the Umini Lai Sutherland (ULS) point. Notice immediately that under a transformation of $c_{j, \mu} \to Uc_{j, \mu}$ we remain invariant at the ULS point. This tells us that the ULS point has global SU$(3)$ symmetry.

 The ground state of the SU(3) point is the Trimer liquid state. \AM{Need to understand this construction better.} 
 
 Also understand this statement: \emph{In general, critical phases of SU($\nu$) symmetric spin chains are stabilised by $\mathbb{Z}_\nu$, the center of SU($\nu$).}
\subsection{Babujian-Takhtajan point}
\subsection{How does the WZW show up in spin S chains?}
Sources: \cite{Affleck:1987ch, itoi1997extended}.

For the case of the BLBQ model, we can show that the critical theory (CFT) at the ULS point is the  SU(3) level 1 WZW model. We can obtain this by perfroming a Hubbard stratanovich transformation to get the euclidean action, and then write down the mean field description to get the IR theory. \AM{Complete this please!} As expected, the CFT central charge adds up correctly (i.e matches the central charge from Yang-Baxter\AM{how to do this?}) after we construct the theory from the lattice model.

The IR gauge-fixed action looks like
\begin{equation}
    S = \int \frac{\d^2 z}{2\pi} \left(\mathcal{L}_{\text{matter}} + \mathcal{L}_{\text{gauge}} + \mathcal{L}_{\text{ghost}} \right)
\end{equation}
where the lagrangians are given by
\begin{equation}
    \mathcal{L}_{\text{matter}} = 2 \bar{\psi}^{\dagger}_{L, \alpha} \bar{\partial} \psi_{L, \alpha} + 2\psi^{\dagger}_{R, \alpha} \partial \psi_{R, \alpha} ,\quad 
    \mathcal{L}_{\text{gauge}} = -\partial \phi \bar{\partial} \phi,\quad  \mathcal{L}_{\text{ghost}} = 2 \bar{\eta} \partial \bar{\epsilon} + 2 \eta \bar{\partial} \epsilon
\end{equation}

\AM{Make this proper}
Little about WZW models.
These are a simple example of solvable(in the sense of yang baxter) RCFTs. In these theories we have a gauge group describing some internal symmetry of the theory but also gets enhanced (extended chiral algebra) by the conformal algebra to form an affine lie algebra (kac moody algebra). General philosophy:more symmetry is good news! the theory will turn out to be simple (because it is highly constrained by the symmetries), i.e has few primary operators and the rest derivable using the algebra.  

It is possible to write the WZW theory in terms of elements of the gauge group, and that will turn out to be purely bosonic. However for certain WZW theories like su(n), so(n) and u(n), one can equivalently describe the theory in terms of multiple species of free (Real) fermions. For example, see the action of the SU(3) level 1 WZW described above for ULS. See Lorenz Ebenhardt's notes/lectures.

A Dubious idea:\AM{and most probably completely false}
We can get SPTs in higher dims by stacking 1d SPTs. Is this somehow connected to the fact that WZW describes the boundary of a chern-simons theory? Because we could have some 1d spin chain which has an SPT phase and has an IR description by a WZW theory, and by stacking them the bulk gets gapped (topological) and is described by a CS theory?
\section{Edge modes and Kitaev chain POV}
There is a direct mapping from the Kitaev chain to the spin 1 models. It allows us to see all the subtleties encountered with respect to edge modes/gapped states in the spin 1 case through a simple ``correspondence" to the majoranas on the kitaev chain. Here is some calculations to work out that may give some insight.\AM{Vaguely remember this discussion, make this precise}

Check this out \cite{verresen2017one}
\begin{enumerate}
    \item Strong pairing to weak pairing transition for spinless vs spinful fermions.
    \item (Related) See the $\frac{t}{U}$ map of Hubabrd model to Heisenberg chain, and see what shows up in the leading order of Pert theory.
\end{enumerate}
\section{Bosonization}
\AM{Move refs to study doc}
% Here is a list of refs
% \begin{enumerate}
%     \item David Tong gauge theory notes
%     \item Senechal review
%     \item Jan von delft review
%     \item Shankar
%     \item Fradkin
%     \item Giamarchi
%     \item E.Miranda
%     \item Nersesyan ICTP
%     \item Eggert 1d quantum wires
%     \item luther peschel paper
%     \item kane lectures on bosonization
%     \item fermi liquids and luttinger liquids -hansson and the other set
%     \item Haldane luttinger liquid theory paper
% \end{enumerate}
Here is a list of topics
\begin{enumerate}
    \item Why does bosonization work? When does it work? What is the role of linearization/1d?
    \item point split vs normal ordering to regulate divergences
    \item Klein factors, why is it important when dealing with multiple fermionic species?
    \item How is continuum limit taken?
    \item Commutators, Correlators of all the fields involved
\end{enumerate}

\subsection{Abelian bosonization, constructive approach}
Goal: Given a \textbf{1D} lattice model of fermions, want to represent the fermion fields $\psi_{\mu}(x)$ at the continuum in terms of bosonic fields $\phi_\mu$ through a relation of the form: $\psi_{\mu} \sim F_{\mu} e^{-\i \phi_{\mu}}$ where $\mu$ is the fermion species index. $F_{\mu}$ is the Klein factor, which \emph{lowers} the number of $\mu$ fermions by one. This relation is an \emph{operator identity} in Fock space.
This is the general philosophy of bosonization, however in the constructive approach, we want to build the bosonic field in the continuum from ``scratch", just using the ingredients from our model on a lattice. (As opposed to the field theory approach, where the existence of the bosonic fields is taken for granted)

Important developments in the constructive side where through the works of Mattis and Lieb, Luther and Peschel, Emery, Haldane.

There are many ways to see why bosonization works. Here is one to start: It turns out that $\psi_{\mu} (x) \ket{0}$ is an eigenstate of the boson annihilation operators $b_{q\mu}$ from which the boson field $\phi_{\mu} (x) = - \sum_{q>0} \sqrt{\frac{2\pi}{qL}}\left(e^{-q(\i x + \frac{a}{2})} b_{q\mu} + \text{h.c.} \right)$ is constructed. This means it must have a \emph{coherent state} representation in terms of the boson creation operators, which is exactly the bosonization relation.

\subsection{Field theory/CFT approach}
\subsection{Non-abelian bosonization}
\section{QFT in 1+1 d}
List of topics
\begin{enumerate}
    \item Luttinger model, luttinger liquid
    \item spin-charge separation
    \item sine-gordon RG analysis (BKT)
    \item Tong 1+1 d
\end{enumerate}
\subsection{Luttinger Model/Liquid}
The idea behind this model is to ``replace" fermi liquid theory in 1 spatial dimension. One dimensional systems exhibit physics that is dramatically different compared to higher dimensions. One peculiarity is that there is no concept of ``single particle" states/excitations. Everything is collective in 1d. The way to handwave this is as follows: In higher dims, you can squeeze past neighbouring species in your way, in 1 dim you can't, there is simply no space! Interactions in 1 dimension therfore always lead to collective behaviour. \emph{All low energy states of a luttinger liquid can be described in terms of free quasiparticle excitations of the massless scalar field.}

One can see the breakdown of fermi liquid theory in many ways. One is to see how many-body perturbation theoretic (diagrams!) treatment of these systems looks like and compare with usual diagrammatics in higher dim. See Dzyaloshinskii Larkin solution for more. \cite{sadovskii2006diagrammatics, giamarchi2003quantum}

Main features: massless/gapless, relativistic scalar fields representing density and phase fluctuations. 
\section{A vaguely stated connection...}
After seeing a collection of different (but related) ideas pop-up in multiple contexts, I wanted to summarise them in a nice way and maybe see if I can atleast give a semblance of ``proof" for these statements. I am quite confident that the way in which it is written down below is completely wrong, but nevertheless I want to keep track of it and try finding counterexamples to disprove every incorrect implication.

\begin{conn}
    Systems with a disordered state $~\longleftrightarrow~$ Exponential Decay of correlation functions 
    $~\longleftrightarrow~$ Gapped spectrum of the Hamiltonian
    $~\longleftrightarrow~$ Existence of short-range interacting/localized/massive excitations
    $~\longleftrightarrow~$ Propagation speed of excitations is less than a ``prescribed speed limit".
    $~\longleftrightarrow~$ System described at some limit by a TQFT + impurity (=quasiparticles) \AM{Check Witten's lec, also in a TQFT there are no propagating modes.}
\end{conn}

Now here is a complementary connection
\begin{conn}
    System at Criticality 
    $~\longleftrightarrow~$ Power law decay of correlation functions
    $~\longleftrightarrow~$ Gapless spectrum of Hamiltonian
    $~\longleftrightarrow~$ Massless excitations/Long Range excitation
    $~\longleftrightarrow~$ Propagation speed of excitation equals speed limit
    $~\longleftrightarrow~$ System described at some limit by a CFT.
\end{conn}
A related question: What sets the speed limit? I guess partially answered by Lieb-Robinson, but want some more intuition. There is also a further relation to chirality of excitations, but I am not able to make a concrete statement yet.

Regarding chirality: Basically, it makes sense to talk about chiral modes on the edge when the bulk is gapped. If you had a gapless bulk then these edge modes are not ``independent" so it does not make sense to talk of the chirality of the modes. Though it is not true that chiral modes are always accompanied by a gapped bulk.\AM{Please find the counterexample!}

A potential counter example to connection 1: Kitaev model (in the context of kitaev spin liquids; with hexagonal lattice in 2+1d) is gapless yet does not have power-law behaviour of spin correlators. They have this weird thing where they vanish sharply beyond a point on the lattice, instead of a smooth decay.\AM{Check this!}
\bibliographystyle{apsrev4-1long}
\bibliography{bibliography}

\end{document}
